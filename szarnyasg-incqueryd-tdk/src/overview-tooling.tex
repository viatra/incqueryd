\section{Tooling}

To be able to focus on the distributed aspects, of the system, we aimed to build \iqd{} on top of \eiq{}'s pattern language (IQPL) and its Rete network generator. Also, \eiq{} has an Eclipse-based user interface for defining and executing queries.

For \iqd{}, we plan to provide the same tooling environment. Also, for the allocation of Rete nodes, we created an Eclipse-based editor and viewer.

% see also http://components.neo4j.org/neo4j/snapshot/apidocs/org/neo4j/graphdb/PropertyContainer.html

%\subsection{Runtime Model-based Dashboard}
%\label{dashboard}

To aid the system's dynamic capabilities, we plan to develop a \emph{runtime model-based dashboard} to monitor the state of \iqd{}'s nodes. Currently, the \iqd{} tooling generates an architecture file (\texttt{arch}), which is used for deploying the distributed pattern matcher.

This file contains the Rete network's layout and its allocation in the cloud (as of now, the latter is defined manually). \iqd{} uses the architecture description for instantiating the Rete network and initializing the middleware (\figref{incqueryd-architecture-dashboard}).

\picTiny{incqueryd-architecture-dashboard}{Architecture of \iqd{} with a runtime dashboard}

To provide live feedback, we will adopt a \emph{live} architecture model. The live model will provide real-time details about the systems' current state, including the local resources on each server, the Rete nodes' memory consumption and so on.

%\section{Tooling}
%\label{tooling}

%\section{Workflow}   

%%%%%%%%%%%%%%%%%%%%%%%%%%%%%%%%%%%%%%%%%%%%%%%%%%%%%%%%%%%%%%%%%%%%%%%%%%%%%%%%%%%%%%%%%%%%%%%%%%%%

\section{Workflow}
\label{workflow}

\pic{incquery-workflow}{\eiq{}'s workflow}

In the following, we will describe the workflow behind the pattern matching process. Starting from a metamodel, an instance model and a graph pattern, we will cover the problem pieces that need to be solved for setting up an incremental, distributed pattern matcher. The workflow is shown on \figref{incquery-workflow}. First, we describe the workflow of \eiq{} and then emphasize the differences in \iqd{}'s.

\subsection{\eiq{}'s workflow}
\label{eiq-workflow}

Based on the \emph{metamodel} and the \emph{query specification}, \eiq{} first constructs a Rete network~\textcircled{1} and deploys it~\textcircled{2}. It loads the model (from the persistent storage) to an \emph{in-memory storage}~\textcircled{3} and traverses it to initialize the Rete network's indexers. The Rete network evaluates the query by processing the incoming tuples~\textcircled{4}. If the modeling application modifies the model through the EMF API, the modifications are propagated the Rete network, hence keeping it in a consistent state~\textcircled{5}. The query results can be retrieved from the Rete network~\textcircled{6}. The modeling application may modify the model and reevaluate the query again.

\subsection{\iqd{}'s workflow}
\label{iqd-workflow}

By design, \iqd{}'s workflow's steps are similar to \eiq{}'s, discussed in \autoref{eiq-workflow}. However, due to the system's distributed nature, they are more difficult to design and implement.

The main differences are the following. In \iqd{}, deploying the Rete network~\textcircled{2} requires the deployment of remote actors (\autoref{akka}) on the servers. Both the Rete indexers and the database are distributed across the cluster. Hence, loading the model and initializing the Rete network needs network communication~\textcircled{3}. The Rete network works using Akka's remote messaging feature. The query results can be retrieved from the Rete network (this may also require network communication)~\textcircled{4}. The database shards can only be accessed through the middleware, which is reponsible for sending notifications to the Rete network's appropriate indexers. After the notifications are processed and the distributed termination algorithm finishes, the Rete network is in a consistent state~\textcircled{5}. The results can be retieved by the client and it may modify the model an reevaluate the query again~\textcircled{6}. 
