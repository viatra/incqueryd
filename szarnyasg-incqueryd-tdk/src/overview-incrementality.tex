\section{Incremental queries, change propagation}
\label{sec:incrementality}

Incremental query evaluation introduces a number of challenges. In the following section, we will describe these and present \iqd{}'s solutions for them.

\subsection{Rete}

The Rete algorithm (\autoref{subsec:rete}) utilizes both indexing and caching to provide fast incremental query evalution.  

\subsubsection{Detailed Rete with an actual instance model}



\figref{rete-routesensor-example-instances}

\myFigure{rete-routesensor-example-instances}{A modification on a \emph{\tb{}} instance model}

\figref{rete-routesensor-example-rete}

\myFigure{rete-routesensor-example-rete}{The Rete net and the partial matches stored in its nodes}





\subsection{Distributed operation}

principles and practice of distributed op


\subsubsection{Principles}

The system should be able to allocate the Rete nodes to different hosts in a cloud computing infrastructure. As the Rete algorithm's change propagation is asynchronous, the system must implement a \emph{termination protocol} to ensure that the query results can be retrieved consistently with the model state after a given transaction (i.e.\ by signaling when the update propagation has been terminated).


\subsubsection{Practice}

In the prototype of \iqd{}, the distributed middleware and Rete network were implemented using Akka (\autoref{subsec:akka}). The communication protocol was built on top of Akka's built-in serialization support.



\subsection{Scalability considerations}
For the storage layer, the most important issue from an incremental query evaluation perspective is that the indexers of the middleware should be filled as quickly as possible. This favors technologies where model sharding can be performed efficiently (i.e.\ with balanced shards in terms of type-instance relationships), and elementary queries (or model graph traversals) can be executed efficiently.

Achieving scalability of the distributed Rete architecture is an equally complex challenge. The overall performance of the system is influenced by a number of factors, including (i) the \emph{layout of the Rete network} (which can be optimized depending on both query and instance model characteristics, e.g.\ to keep the resource requirement of intermediate join operations to a minimum), (ii) the \emph{allocation} of Rete nodes to host computers (e.g.\ to optimize local resource usage, or to minimize the amount of remote network communication), and (iii) \emph{dynamic adaptability} to changing conditions (e.g.\ when the model size and thus query result size grows rapidly, the Rete network may require dynamic reallocation or node sharding due to local resource limitations).

