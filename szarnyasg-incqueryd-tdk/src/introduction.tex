\chapter{Introduction}
\label{chap:introduction}

\section{Context}

Nowadays, model-driven software engineering (MDE) plays an important role in the development processes of critical embedded systems. Advanced modeling tools provide support for a wide range of development tasks such as requirements and traceability management, system modeling, early design validation, automated code generation, model-based testing and other validation and verification tasks. 

Models representing sensor data, reverse engineered software models (e.g.\ abstract syntax trees of existing source code) and geospatial models can contain well over $10^9$ modeling elements \cite{Scheidgen12}. The dramatic increase in complexity is also affecting critical embedded systems in recent years. Modeling toolchains are facing scalability challenges as the size of design models constantly increases, and automated tool features become more sophisticated.

\section{Problem Statement and Requirements}

Many scalability issues can be addressed by improving query performance. \emph{Incremental evaluation} of model queries aims to reduce query response time by limiting the impact of model modifications to query result calculation. Such algorithms work by either (i) building a cache of interim query results and keeping it up-to-date as models change (e.g.\ \eiq{}~\cite{models10}) or (ii) applying impact analysis techniques and re-evaluating queries only in contexts that are affected by a change (e.g.\ the Eclipse OCL Impact Analyzer~\cite{OCLIA}). This technique has been proven to improve performance dramatically in several scenarios (e.g.\ on-the-fly well-formedness validation or model synchronization), at the cost of increasing memory consumption. Unfortunately, this overhead is combined with the increase in model sizes due to in-memory representation (found in state-of-the-art frameworks such as EMF~\cite{EMF}). Since single-computer heaps cannot grow arbitrarily (as response times degrade drastically due to garbage collection problems), memory consumption is the most significant scalability limitation.% of the single workstation architecture.

An alternative approach to tackling MDE scalability issues is to make use of advances in persistence technology. As the majority of model-based tools uses a graph-oriented data model, recent results of the NoSQL and Linked Data movement~\cite{neo4j,openvirtuoso,sesame} are straightforward candidates for adaptation to MDE purposes. Unfortunately, this idea poses difficult conceptual and technological challenges: (i) property graph databases lack strong metamodeling support and their query features are simplistic compared to MDE needs, and (ii) the underlying data representation format of semantic databases (RDF~\cite{website:rdf_standard}) has crucial conceptual and technological differences to traditional metamodeling languages such as Ecore~\cite{EMF}. Additionally, while there are initial efforts to overcome the mapping issues between the MDE and Linked Data worlds~\cite{hillairet2008bridging}, even the most sophisticated NoSQL storage technologies lack efficient and mature support for executing expressive queries \emph{incrementally}.

\section{Objectives and Contributions}

We aim to address these challenges by adapting incremental graph search techniques from \eiq{} to the cloud infrastructure. We introduce \iqd{}, a prototype system based on a distributed Rete network~\cite{Forgy} that can scale up from a single-workstation tool to a cluster to handle very large models and complex queries efficiently.
 
The main contributions of this report is a \emph{novel architecture} for a distributed, incremental query engine. We developed a \emph{working prototype} and designed a \emph{benchmark environment} to evaluate the scalability characteristics of the system. 

We conducted benchmarks with different storage backends and query engines. The analysis of the results confirmed the feasibility of the approach.

\section{Motivations for Horizontal Scaling}

\eiq{} is proven to be effecient for incremental query evaluation on small to medium-sized models (in the order of magnitude of $10^6$). However, model-driven engineering challenges often present large models, with $10^9$ or more elements \cite{Scheidgen12}.

Due to the Rete algorithm's memory consumption (\autoref{rete}), \eiq{} cannot handle large models efficiently \cite{models10}. A trivial solution would be to use \emph{vertical scaling}, i.e.\ putting more memory in the workstation. Unfortunately, this approach is not feasible due to nature of Java's memory management. The Garbage Collector (GC) cannot handle heap sizes larger than 10~GB efficiently, thus introducing long pauses in the application \cite{Azul}. 

This problem is well-known in the Java community. There are alternative Java Virtual Machines (JVMs) with specialized Garbage Collectors, like Azul Systems' JVM. However, the Azul JVM is a proprietary product and has specific hardware requirements. Also, this does not solve the scaling problem entirely -- the model size is still limited by the total amount memory in a single computer.

Instead of vertical scaling, we decided to opt for \emph{horizontal scaling}. As described in \autoref{nosql}, distributed non-relational databases have been gaining momentum in the last years. For persisting models, we inspected graph databases, but found that their query layer does not scale well in a distributed environment (\autoref{evaluation-results}). This is a serious problem for MDE, where the queries are typically more complex than in traditional, transactional database management.

Therefore, we decided to design a stand-alone distributed, scalable query engine based on NoSQL and semantic web databases.


% miben haladja meg a state of the artot
% scalability
% open-source technologiak lekerdezeseinel gyorsabbak vagyunk

\section{Structure of the Report}
 
The report is structured as follows. 
\autoref{chap:background-technologies} introduces the background technologies and the motivation for building a distributed, incremental graph pattern matcher. \autoref{chap:overview} provides an overview of current a single-node incremental pattern macher, \eiq{}. 
\autoref{chap:evaluation} shows an initial performance evaluation in the context of on-the-fly well-formedness validation of software design models. 
\autoref{chap:related-work} discusses the related work. \autoref{chap:conclusions} concludes the report and presents our future plans. 

