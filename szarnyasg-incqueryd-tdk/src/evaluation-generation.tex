\section{Generation of models}

We created a property graph generator project based on the previous \tb{} generators. The generator creates a graph in an embedded Neo4j database and uses the Blueprints library's \texttt{GraphMLWriter} class to save to a \graphml{} file~\cite{Blueprints}.

%\chapter{Evaluation}
%\label{sec:evaluation}

% 1 hasab + 1 abra helyed van, helytakarekosan irj, keruld az itemize-okat

We implemented \iqd{} as an initial prototype to evaluate the feasibility of the approach, and to experiment with various optimization possibilities. As the storage, we used the popular graph database Neo4j~\cite{neo4j} featuring automatic indexes and two core query technologies (Gremlin and Cypher) that were used as a low-level model access interface by our middleware layer.
The prototype of the distributed middleware and Rete network were implemented in Java using Akka~\cite{akka}, the Scala-based toolkit for building applications based on the Actor model, since it is well-suited for asynchronous applications. The communication protocol was built on top of Akka's built-in serialization support.


%TODO mit merunk? model manipulacios muveletek es query kiertekeles valaszido
%adott: query def, modell manipulacios szekvencia

% 4 fazisban az alabbiak szerint TODO
%generalt: novekvo meretu modellek (hogyan generaltuk, mi a query-k eredmenyhalmaz meretenek viszonya a modellhez?, 

%mi az elosztott modell sajatossaga/limitacioja (nincsenek keresztelek))

%mert: lekerdezesek, ill. manip. tranzakcio lefutasi ideje
