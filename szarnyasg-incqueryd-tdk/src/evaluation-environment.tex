\section{Benchmark Environment}
\label{sec:benchmark-environment}

describe the benchmark envirnment


Big MDE article \cite{Izso:2013:IIG:2487766.2487772}

\subsection{Benchmark Setup}

\pic{benchmark-scenario}{The benchmark scenario}

distributed the Rete net as shown on \figref{rete-routesensor-distributed-layout}

\pic{rete-routesensor-distributed-layout}{The layout of the distributed Rete net}

\subsection{Hardware and Software Ecosystem}

As the testbed, we deployed our system to a private cloud consisting of 4 virtual machines on separate host computers. 

\subsubsection{Hardware}
 
For Titan and 4store, each virtual machine used dual 2.5 GHz Intel Xeon L5420 CPUs with 8 GBs of RAM, running on Ubuntu 12.10 64-bit.

For Neo4j, each virtual machine had the same parameters but twice as much, 16 GBs of RAM. 

\subsubsection{Software}

We used the following technologies:

\begin{itemize}
  \item Ubuntu 12.10 64-bit
  \item Oracle Java 7 64-bit
  \item 4store 1.1.5
  \item Titan 0.3.2
  \item Faunus 0.3.2
  \item Hadoop 1.1.2
  \item Cassandra 1.2.2
  \item Neo4j 1.8
  \item Blueprints 2.3.0
  \item Akka 2.1.2
  \item Eclipse 4.3 (Kepler)
\end{itemize}

\subsection{Benchmark Methodology}

five runs

some load on servers, so used the \emph{minimum} of the results

\subsection{Data Collection}

\tb{}'s utility classes

R script \cite{RProject} developed by Benedek Izsó 

\subsection{Data Processing Tools}

To compare the performance characteristics of \iqd{} to a traditional case, we defined two scenarios. 

The \textit{batch} scenario uses only 4store to manage models and evaluate the queries (depicted as \textcircled{1} in \figref{architecture}). This serves as a baseline for the \textit{incremental} scenario, which uses \iqd{} (shown as \textcircled{2} in \figref{architecture}). For these initial experiments, the layout and allocation of the Rete network was determined manually. 
