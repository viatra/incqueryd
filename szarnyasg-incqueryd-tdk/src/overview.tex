\chapter{Overview}
\label{chap:overview}

The primary goal of \iqd{} is to provide a scalable architecture for executing incremental queries over large models. Our approach is based on the following foundations: (i) a distributed model storage system that (ii) supports a graph-oriented data representation format, and (iii) a graph query language (adapted from the \eiq{} framework). The novel contribution of this report is an architecture and an implementation that consists of a (i) distributed model management middleware, and a (ii) distributed and stateful pattern matcher network based on the Rete algorithm.

\iqd{} provides incremental query execution by \emph{indexing model contents} and \emph{capturing model manipulation operations} in the middleware layer, and \emph{propagating change tokens} along the pattern matcher network to \emph{produce query results and query result changes} (corresponding to model manipulation transactions) efficiently. As the primary sources of memory consumption, i.e.\ both the indexing and intermediate Rete nodes can be distributed in a cloud infrastructure, the system is expected to scale well beyond the limitations of the traditional single workstation setup.

\section{Case study: \tb{}}
\label{sec:trainbenchmark}

Due to both confidentiality and technical reasons, it's difficult to obtain real-world industrial models and queries. Also, using confidential data sets hamstrings the reproducability of the conducted benchmarks. Therefore, we used an artificial data set which mimics real-world models.  

In the following section we present the \textit{\tb{}}, a benchmark scheme and environment. The \tb{} was designed and implemented by Benedek Izsó, István Ráth and Zoltán Szatmári~\cite{Izso:2012:ODD:2428516.2428523}. The original goal of the \tb{} was to compare various (preferably incremental) query engines' performance to \eiq's. For \iqd, we used a slightly modified version of the \tb{}.

\subsection{Domain}

\myFigure{trainbenchmark-metamodel}{The EMF metamodel of the \tb{}}

The \tb{} models is an imaginary railroad network. The network is composed of typical railroad items, including signals, segments, switches and sensors. The complete EMF metamodel of the \tb{} is shown on \figref{trainbenchmark-metamodel}. 

The \textit{generator} project of \tb{} is capable of generating railroad instance models of different sizes. It's capable of generating models in different formats, including EMF, OWL, RDF and SQL. 

For Neo4j (\autoref{subsec:neo4j}) and Titan (\autoref{subsec:titan}), we expanded the generator with a module that can generate property graphs based on the \tb{}'s metamodel. It supports the GraphML~\cite{GraphML}, the Blueprints GraphSON~\cite{BlueprintsGraphSON} and the Faunus GraphSON~\cite{FaunusGraphSON} output formats.

% szerintem felesleges a scenariokrol beszelni, mert csak a UserScenarioval foglalkoztunk -- SzG
% The \tb{} defines two scenarios:
% \begin{description}
%   \item[UserScenario] This scenario simulates a user sitting in front of her workstation and modifying the model in small steps.
%   \item[XFormScenario] This scenario simulates a software running automated transformations on the model.
% \end{description}

\subsection{Queries}

The \tb{} consists of queries that resemble a typical MDE workload. In general, MDE queries are more complex than those used in traditional databases. They often define large patterns with multiple join operations. The \tb{}'s queries look for violations of well-formedness constraints in the model. Although the \tb{} defines four different queries, in this report, we only discuss the \textit{RouteSensor} query in detail.

\subsubsection{RouteSensor}

\myFigure{routesensor-pattern}{Graphical representation of the \emph{RouteSensor} query's pattern. The dashed red arrow defines a negative condition.}

The \textit{RouteSensor} query looks for \textit{Sensor}s that are connected to a \textit{Switch}, but the sensor and the switch are \emph{not} connected to the same \textit{Route}. The graphical representation of the RouteSensor query is shown on \figref{routesensor}. The RouteSensor query  IQPL (\autoref{subsec:eiq}).\footnote{Note that the two queries are slightly different: the SPARQL query returns only a set of sensors, while the IQPL query returns a set of (Sensor, Switch, SwitchPosition, Route) tuples. TODO explain.}

\begin{lstlisting}[caption=The \emph{RouteSensor} query in IQPL, label=lst:routesensor-iqpl]
package hu.bme.mit.train.constraintcheck.incquery

import "http://www.semanticweb.org/ontologies/2011/1/TrainRequirementOntology.owl" 

pattern routeSensor(Sen, Sw, Sp, R) = {
	Route(R);
	SwitchPosition(Sp);
	Switch(Sw);
	Sensor(Sen);
	
	Route.Route_switchPosition(R, Sp);
	SwitchPosition.SwitchPosition_switch(Sp, Sw);
	Trackelement.TrackElement_sensor(Sw, Sen);
	
	neg find head(Sen, R);	
}

pattern head(Sen, R) = {
	Route.Route_routeDefinition(R, Sen);
}
\end{lstlisting}



\lstset{language=SQL,morekeywords={PREFIX,FILTER}} %,java,rdf,rdfs,url,owl,base}}

\begin{lstlisting}[caption=The \emph{RouteSensor} query in SPARQL, label=lst:routesensor-sparql]
PREFIX base: <http://www.semanticweb.org/ontologies/2011/1/TrainRequirementOntology.owl#>
PREFIX rdfs: <http://www.w3.org/2000/01/rdf-schema#>
PREFIX owl:  <http://www.w3.org/2002/07/owl#>
PREFIX rdf:  <http://www.w3.org/1999/02/22-rdf-syntax-ns#>

SELECT DISTINCT ?xSensor
WHERE
{
    ?xRoute rdf:type base:Route .
    ?xSwitchPosition rdf:type base:SwitchPosition .
    ?xSwitch rdf:type base:Switch .
    ?xSensor rdf:type base:Sensor .
    ?xRoute base:Route_switchPosition ?xSwitchPosition .
    ?xSwitchPosition base:SwitchPosition_switch ?xSwitch .
    ?xSwitch base:TrackElement_sensor ?xSensor .

    FILTER NOT EXISTS {
        ?xRoute ?Route_routeDefinition ?xSensor .
    } .
}
\end{lstlisting}






% The Cypher implementation of the RouteSensor query is shown on %\autoref{lst:cypher-routesensor}
% 
% \begin{lstlisting}[caption=Cyper query for the RouteSensor test case, label=lst:cypher-routesensor, breaklines=true]
% START sensor=node:node_auto_index(type="Sensor")
% MATCH sensor-[:TRACKELEMENT_SENSOR]-switch-[:SWITCHPOSITION_SWITCH]-switchPosition-[:ROUTE_SWITCHPOSITION]-route-[r?:ROUTE_ROUTEDEFINITION]-sensor
% WHERE r IS NULL
% RETURN sensor
% \end{lstlisting}


\section{Architecture overview}
\label{sec:architecture}

In the following, we will cover the challenges for building an incremental pattern matcher and present the architecture of \iqd{}.


%\subsection{Initialization and indexing}
%\label{subsec:indexing}
%In the following section we will cover the challenges that arise around the indexing and initialization of \iqd{}.

\subsection{Indexing}
%\label{subsec:indexing}

Indexing is a common technique for decreasing the execution time of database queries. In MDE, \emph{model indexing} is the key to high performance model queries. As MDE primarily uses a metamodeling infrastructure, the \iqd{} middleware maintains type-instance indexes so that all instances of a given type (both edges and graph nodes) can be enumerated quickly. These indexers form the bottom layer of the Rete production network. 

\subsection{Graph-like data manipulation}

\iqd{}'s middleware exposes an API that provides methods to manipulate the graph. By allowing graph-like data manipulation we allow the user to focus on the domain-specific challenges, thus increasing her productivity. The middleware translates the user's operation and forwards it to the underlying data storage (e.g.\ SPARQL queries for 4store and Gremlin queries for Titan).

\subsubsection{Data representation}

Conceptually, the architecture of \iqd{} allows the usage of a wide scale of model representation formats. Our prototype has been evaluated in the context of the \emph{property graph} and the \emph{RDF} data model, but other mainstream metamodeling and knowledge representation languages such as relational databases' SQL dumps and Ecore~\cite{EMF} could be supported, as long as they can be mapped to an efficient and distributed storage backend (e.g.\ triplestores, key-value stores or column-family databases).

To support different data models, we only have to supply the appropriate connector class to \iqd{}'s middleware. The current implementation supports 4store, Neo4j and Titan. % Ertelemszeruen Neo4j-t a mostani verzioba nem hoztam at 100%-ig 

\subsection{Distributed storage layer}

For the storage layer, the most important issue from an incremental query evaluation perspective is that the indexers of the middleware should be filled as quickly as possible. This favors technologies where model sharding can be performed efficiently (i.e.\ with balanced shards in terms of type-instance relationships), and elementary queries 
can be executed efficiently.

\subsection{Notification mechanisms}

% -- The Rete algorithm defines an asynchronous network of communicating nodes. This is essentially a dataflow network, with two types of nodes. Change notification objects (\emph{tokens}) are propagated to intermediate \emph{worker nodes} that perform operations, like filtering tokens based on constant expressions and performing join or antijoin operations based on their contents. The worker nodes store partial (interim) query results in their own memory. In contrast, \emph{production nodes} are terminators that provide an interface for fetching query results and also their changes (\emph{deltas}). Connections between nodes can be \emph{local} (within one host) or \emph{remote} (when two Rete nodes are allocated to different hosts).

\emph{Model change notifications} are required by incremental query evaluation, thus model changes are captured and their effects propagated in the form of \emph{notification objects} (NOs). The notifications generate \emph{tokens} that keep the Rete network's state consistent with the model. \iqd{}'s middleware layer facilitates notifications by providing a facade for model manipulation operations.

\figref{incqueryd-architecture}

\subsubsection{Notification in distributed database management systems}

While relational databases usually provide \emph{trigger}s for generating notifications, most triplestores and graph databases lack this feature. Among our primary database backends, 4store provides no triggers at all. Titan and Neo4j incorporate Blueprints, which provides an \texttt{EventGraph} class capable of generating notification events, but the events are only propagated in a single JVM (Java Virtual Machine). Implementing distributed notifications would require us to extend the \texttt{EventGraph} class and use a messaging framework. This is subject to future work (see \autoref{sec:future-work}). 

Because the lack of support for distributed notifications, in \iqd{}'s current implementation, notifications are controlled by the middleware. The notification messages are propagated through the Rete network via the Akka messaging framework. 

%\subsection{Incremental queries and change propagation}
%\label{subsec:incrementality}

%Distributed incremental query evaluation introduces a number of challenges. In the following section, we will describe these and present \iqd{}'s solutions for them.






\subsection{Incremental query evaluation}
\label{subsec:incrementality}

%Kicsit más logika alapján csinálnám ezt.
%Én előre venném az inkrementális (gráf)lekérdezés, mint ötlet/technika bemutatását, utána mesélném el a Retét, mint ennek egy megvalósítását, és csak nagyon röviden említeném az egyéb alternatívákat (főleg arra koncentrálva, hogy a Retéhez képest mi a jellegzetes különbségük).

Some queries, e.g.\ well-formedness constraints in MDE are evaluated many times, while the data sets they are evaluated on only changes to a small degree. In these cases, the idea of incremental query evaluation arises naturally: to speed up queries, we should not start the evaluation all over again. Instead, we should rely on the (partial) results derived during the previous executions of the query and process the changes that occured.
 
In practice, incremental query evaluation algorithms typically use data structures for caching the interim results. This  means that they usually consume more memory, in other words, they trade memory consumption for execution speed. This approach, called \emph{space--time tradeoff}, is well-known and widely used in computer science.

In the following, we provide an overview of the Rete algorithm, which forms the theoretical basis of \eiq{} and \iqd{}.

\subsubsection{Rete in general}
\label{subsubsec:rete}

\iqd{} is based on the Rete algorithm, which provides a propagation network for incremental graph pattern matching. The algorithm was originally created by Charles Forgy~\cite{Forgy} for rule-based expert systems. Gábor Bergmann adapted the algorithm for EMF models and added many tweaks and improvements to it~\cite{BergmannRete}.

The Rete algorithm defines an asynchronous network of communicating nodes. This is essentially a dataflow network, with two types of nodes. Change notification objects (\emph{tokens}) are propagated to intermediate \emph{worker nodes} that perform operations, like filtering tokens based on constant expressions and performing join or antijoin operations based on their contents. The worker nodes store partial (interim) query results in their own memory. In contrast, \emph{production nodes} are terminators that provide an interface for fetching query results and also their changes (\emph{deltas}). Connections between nodes can be \emph{local} (within one host) or \emph{remote} (when two Rete nodes are allocated to different hosts).

\subsubsection{Similar algorithms}

Along the original Rete algorithm, many algorithms were developed for incremental pattern matching. Rete itself has improved versions (Rete II, Rete III, Rete-NT), however, unlike the original algorithm, these are not publicly available. 

Other novel algorithms include TREAT \cite{Miranker:1991:OPT:627280.627434}, which aims at minimizing memory using only indexers and dropping partial results, while having the same algorithmic complexity as Rete. Another candidate is the LEAPS \cite{Batory:1994:LA:899216} algorithm, which is claimed to provide better space--time complexity. However, we found that even LEAPS is difficult to understand and implement even on a single workstation, not to mention the distributed case. 

Because the Rete algorithm is well-known and well-understood by the \eiq{} team, we decided to build \iqd{} on the same foundation. Experimenting with improved versions or alternative approaches is subject to future work.

\subsection{\iqd{}'s architecture}

\iqd{}'s architecture consists of three layers: the storage layer, the middleware and the production network. 
The \emph{storage layer} is a distributed database which is responsible for persisting the graph. 
The client application communicates with the \emph{middleware}. The middleware provides a unified API for accessing the database. It also sends change notifications to the production network and retrieves the query results from the production network. 
The \emph{production network} is implemented with a distributed Rete net which provides incremental query evaluation. 

\myFigure{incqueryd-architecture}{\iqd{}'s architecture on a four-node cluster}

The \iqd{} architecture in a four-node cluster configuration is shown in \figref{incqueryd-architecture}.


\subsection{Deployment and configuration}
\label{subsec:deployment-configuration}

Deploying, configuring and operating a distributed pattern matcher is a complex task. In the following, we will break down this task to smaller steps and present our tools for solving them.

\subsubsection{Degrees of freedom}

The deployment and configuration of a distributed pattern matcher involves many degrees of freedom:

\begin{itemize}
  \item We may choose different database implementations due to the \iqd{}'s backend-agnostic nature. Until now, we experimented with property graph databases (Neo4j, Titan) and triplestores (4store).
  \item We may use different database sharding strategies (e.g.\ random partitioners or more sophisticated sharding methods based on domain-specific knowledge).
  \item We may choose different strategies to allocate the Rete nodes. Note that in theory, this is orthogonal to the database's sharding strategy. However, we expect that keeping the Rete network's type indexer nodes and the instances of the given type on the same server would improve the speed of the initialization and modification tasks significantly.
\end{itemize}
 
\subsection{Workflow}
\label{subsec:workflow}

In the following part, we will describe the workflow behind the pattern matching process. Starting from a metamodel, an instance model and a graph pattern, we will cover the problem pieces that need to be solved for setting up an incremental, distributed pattern matcher. The workflow is shown on \figref{recipe}.

\myFigureSmall{recipe}{The workflow of \eiq{} (blue) and \iqd{} (green)}

\subsubsection{Analyze the metamodel and the query}

\paragraph{Task.} First, we determine the constraints defined by the query pattern. The matches satisfying these constraints will define the results of the query.

\paragraph{Implementation.} The pattern is defined in an IQPL (\iq{} Pattern Language) text file. Using Xtext~\cite{Xtext}, an Eclipse-based framework for creating domain-specific languages, the textual representation of the pattern is parsed to an EMF model. Based on the EMF model's pattern and the metamodel, a constraint network called \textit{PSystem} (short for \textit{Pattern System}) is generated. 

\subsubsection{Build a Rete layout}

\paragraph{Task.} To allow incremental query evaluation, we create a Rete net based on the constraints derived from the query.

\paragraph{Implementation.} As we mentioned earlier, we aim to reuse as much of \eiq{}'s existing code base as possible. As part of this attempt, we introduced the concept of \textit{Rete recipe}s which define the layout of a Rete network.    

\subsubsection{Allocate the Rete network in the cloud's nodes} 

\paragraph{Task.} Because of its single workstation nature, \eiq{} simply unfolds the Rete net based on the derived Rete recipe. At the same time, \iqd{} operates in a distributed environment where local resource exhaustion, network latency and throughput are critical aspects. 

\paragraph{Implementation.} Currently, the allocation of the Rete nodes is done manually. To address this limitation, we plan to utilize CSP (Constraint Satisfaction Problem) solvers, or dynamic techniques like DSE (Design Space Exploration)~\cite{DSE11}. 

\subsubsection{Bootstrap the system}

\paragraph{Task.} Based on the Rete network's allocation, we have to deploy the Rete nodes in the distributed systems. After the successful deployment, the Rete network has to be initialized. Due to the Rete algorithm's asynchronous nature, it uses a termination protocol to signal when the data processing is finished. 

\paragraph{Implementation.} In \iqd{}'s prototype, both the bootstrapping and the Rete network's operation is carried out automatically. The Akka actors representing the Rete nodes are deployed and initiated using Akka's \textit{programmatic remote deployment} feature. For signalling the end of data processing, an asynchronous termination protocol was implemented. 


\subsection{Termination protocol}



\section{Middleware}
\label{sec:middleware}

\subsection{Indexing}


\subsection{Graph-like data manipulation}


\subsection{Notification mechanisms}



\emph{Storage and middleware.}\label{storage_and_middleware}

The proposed system is based on a distributed database management system.

In recent years, along long standing relational database management systems, dozens of new database systems sprung to life. This systems are often called NoSQL (short for not only SQL) databases.
These systems are often specialized to serve a specific aspect of Web 2.0 applications. To do so, they provide a non-relational data model and weaker consistency guarantees, but offer higher availability and better scalability.

In contrast to a traditional setup, where the distributed model repository (consisting of four shards in the example) is accessed on a per-node basis by a model manipulation transaction (such as a model transformation benchmark, depicted as $T_{BM}$ in \figref{architecture}), \iqd{} provides a middleware layer that offers three core services (shown in green in \figref{architecture}).
In {{\em distributed model management}}, the primary task is to provide a \emph{surrogate key} mechanism so that each model element in the entire distributed repository can be uniquely identified, and located within storage shards.
{{\em Model indexing}} is the key to high performance model queries. As MDE primarily uses a metamodeling infrastructure, the \iqd{} middleware maintains type-instance indexes so that all instances of a given type (both edges and graph nodes) can be enumerated quickly.
Finally, {{\em model change notifications}} are required by incremental query evaluation, thus model changes are captured and their effects propagated in the form of \emph{notification objects} (NOs). The middleware layer achieves this by providing a facade for model manipulation operations. 

Conceptually, the architecture of \iqd{} allows the usage of a wide scale of model representation formats. Our first prototype has been evaluated in the context of a low abstraction level \emph{property graph}~\cite{DBLP:journals/corr/abs-1006-2361} data model, but other mainstream metamodeling and knowledge representation languages such as Ecore~\cite{EMF} and RDF~\cite{website:rdf_standard} could be supported, as long as they can be mapped to an efficient and distributed storage backend (like key-value stores or column-family databases).


\section{Incremental queries, change propagation}



\subsection{Rete}


\subsubsection{Detailed Rete with an actual instance model}


\myFigure{rete-routesensor-example-instances}{An instance model of the \tb's metamodel}

\myFigure{rete-routesensor-example-rete}{The Rete net and the partial matches stored in its nodes}



\subsection{Distribution}


\subsubsection{Principles}


\subsubsection{Practice (transparent framework: Akka)}





\emph{Distributed pattern matcher.}\label{distributed_pattern_matcher}
On top of the middleware, \iqd{} constructs a distributed and asynchronous network of communicating nodes that implement the Rete~\cite{Forgy} algorithm (shown within the dashed region in \autoref{fig:architecture}). This layer is essentially a dataflow network, with two types of nodes. Change notification objects (tokens) are propagated to intermediate \emph{worker nodes} that perform operations (like filtering tokens based on constant expressions, or performing join or antijoin operations based on their contents) and
store partial (interim) query results in their own memory. In contrast, \emph{production nodes} are terminators that provide an interface for fetching query results and also their changes. Connections between nodes can be \emph{local} (within one host) or \emph{remote} (when two Rete nodes are allocated to different hosts). It is important to emphasize that the database shards and Rete nodes are two distinct levels of distribution that do not directly depend upon each other.

% \begin{figure}[!h]
% \begin{center}
% \begin{tabular}{cc}
% \begin{minipage}{0.35\columnwidth}
% \begin{lstlisting}
% pattern test(
%   V1:Type1, V2:Type2,
%   V3:Type3, V4:Type4) {
%   Type1.edgeType1(V1, V2);
%   // join 1
%   Type2.edgeType2(V2, V3);
%   // join 2
%   Type3.edgeType3(V3, V4);
%   // antijoin
%   neg find anti(V4, V1);
% }
% pattern anti(V4, V1) {
%  Type1.edgeType4(V4, V1);
% }
% \end{lstlisting}
% \end{minipage}
% \hspace{0.5cm}
% \begin{minipage}{0.45\columnwidth}
%   \includegraphics[width=\textwidth]{figures/patterndef}
% \end{minipage}
% \end{tabular}
% \end{center}
% \caption{Example graph query}
% \label{fig:patterndef}
% \end{figure}



\subsection{Scalability considerations}
For the storage layer, the most important issue from an incremental query evaluation perspective is that the indexers of the middleware should be filled as quickly as possible. This favors technologies where model sharding can be performed efficiently (i.e.\ with balanced shards in terms of type-instance relationships), and elementary queries (or model graph traversals) can be executed efficiently.

Achieving scalability of the distributed Rete architecture is an equally complex challenge. The overall performance of the system is influenced by a number of factors, including (i) the \emph{layout of the Rete network} (which can be optimized depending on both query and instance model characteristics, e.g.\ to keep the resource requirement of intermediate join operations to a minimum), (ii) the \emph{allocation} of Rete nodes to host computers (e.g.\ to optimize local resource usage, or to minimize the amount of remote network communication), and (iii) \emph{dynamic adaptability} to changing conditions (e.g.\ when the model size and thus query result size grows rapidly, the Rete network may require dynamic reallocation or node sharding due to local resource limitations).


\section{Deployment, configuration}
\label{sec:deployment-configuration}

\subsection{Tooling}

reuse of \eiq's components

Eclipse-based editor

\myFigure{recipe}{The workflow of \eiq\ (blue) and \iqd\ (green)}


\subsection{Degrees of freedom}

database sharding, allocation of rete nodes -- orthogonal

choosing different dbs

\subsubsection{Different database implementations}

diffent storage backends are supported



\subsection{Workflow}  %Problem pieces}

workflow from an IQPL pattern to the working pattern matcher 

\subsubsection{Analyze the model and the query}

IQPL -> EMF model

\paragraph{Task.}

\paragraph{Implementation.} IQPL, EMF model, P-System 

\subsubsection{Build a Rete layout}

EMF model -> PSystem -> recipe -> rete layout

\paragraph{Task.}  

\paragraph{Implementation.} \eiq{}'s implementation 

\subsubsection{Allocate the Rete network in the cloud's nodes}

distribution, latency, throughput, .. 

\paragraph{Task.} 

\paragraph{Implementation.} CSP, DSE~\cite{DSE11}

\subsubsection{Bootstrap the system}

deploying actors, initiating the Rete net's processing workflow

\paragraph{Task.} 

\paragraph{Implementation.} Automatic bootstrapping and operation 



\myFigure{incqueryd-tooling-tree-editor}{The editor in \iqd's tooling}

\myFigure{incqueryd-tooling-yfiles-viewer}{The viewer in \iqd's tooling}


\section{Elaboration of the example}
\label{sec:elaboration}

We use the \textit{RouteSensor} query as our example. The query is shown as a graph pattern definition on \lstref{routesensor-iqpl} and visualized on \figref{routesensor-pattern}. Queries like this are typical in MDE applications (such as well-formedness validation or complex model transformations).

\subsection{Workflow}

Following the workflow defined in \autoref{subsec:workflow}, we will cover the actual steps for deploying and operating a distributed pattern matcher for the \textit{RouteSensor} query.

\subsubsection{Analyze the metamodel and the query}

The metamodel is shown on \figref{trainbenchmark-metamodel}. Using \eiq{}'s tooling, the textual representation (\texttt{routeSensor.arch}, see \lstref{routesensor-iqpl}) is analyzed and parsed to an EMF model (\figref{eiq-model}).

\myFigureSmall{eiq-model}{The EMF model generated from the pattern}

\subsubsection{Build a Rete layout}

Based on the query's EMF model, \eiq{}'s tooling builds PSystem and creates a Rete layout, that quarantees the satisfaction of the constraints. The Rete layout is shown on \figref{rete-routesensor-example-rete}. 

\subsubsection{Allocate the Rete network in the cloud's nodes} 

In \iqd{}'s current implementation, the Rete recipe's nodes are allocated manually on the cloud servers (called \textit{Machine}s). The allocation is currently defined in an architecture file (e.g.\ \texttt{routeSensor.arch}). The Rete nodes are associated with the machines with \textit{infrastructure mapping} edges.

\iqd{}'s tooling currently provides an Eclipse-based tree editor to define machines and the infrastructure mapping edges (\figref{incqueryd-tooling-tree-editor}).

\myFigureSmall{incqueryd-tooling-tree-editor}{The tree editor in \iqd's tooling}

The tooling is capable of visualizing the Rete network and its mapping to the machines (see \figref{incqueryd-tooling-yfiles-viewer})

\myFigureSmall{incqueryd-tooling-yfiles-viewer}{The yFiles viewer in \iqd's tooling}

\subsubsection{Bootstrap the system}

\iqd{}'s current implementation, the distributed system is initiated with a Bash script which launches the Akka microkernel on the appropriate nodes. The Akka actors representing the Rete network's nodes are deployed automatically by the \iqd{} \textit{Coordinator} node.



