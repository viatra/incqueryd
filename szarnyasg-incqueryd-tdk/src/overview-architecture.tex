\section{Architecture overview}
\label{sec:architecture}

In the following, we will cover the challenges for building an incremental pattern matcher and present the architecture of \iqd{}.


%\subsection{Initialization and indexing}
%\label{subsec:indexing}
%In the following section we will cover the challenges that arise around the indexing and initialization of \iqd{}.

\subsection{Indexing}
%\label{subsec:indexing}

Indexing is a common technique for decreasing the execution time of database queries. In MDE, \emph{model indexing} is the key to high performance model queries. As MDE primarily uses a metamodeling infrastructure, the \iqd{} middleware maintains type-instance indexes so that all instances of a given type (both edges and graph nodes) can be enumerated quickly. These indexers form the bottom layer of the Rete production network. 

\subsection{Graph-like data manipulation}

\iqd{}'s middleware exposes an API that provides methods to manipulate the graph. By allowing graph-like data manipulation we allow the user to focus on the domain-specific challenges, thus increasing her productivity. The middleware translates the user's operation and forwards it to the underlying data storage (e.g.\ SPARQL queries for 4store and Gremlin queries for Titan).

\subsubsection{Data representation}

Conceptually, the architecture of \iqd{} allows the usage of a wide scale of model representation formats. Our prototype has been evaluated in the context of the \emph{property graph} and the \emph{RDF} data model, but other mainstream metamodeling and knowledge representation languages such as relational databases' SQL dumps and Ecore~\cite{EMF} could be supported, as long as they can be mapped to an efficient and distributed storage backend (e.g.\ triplestores, key-value stores or column-family databases).

To support different data models, we only have to supply the appropriate connector class to \iqd{}'s middleware. The current implementation supports 4store, Neo4j and Titan. % Ertelemszeruen Neo4j-t a mostani verzioba nem hoztam at 100%-ig 

\subsection{Distributed storage layer}

For the storage layer, the most important issue from an incremental query evaluation perspective is that the indexers of the middleware should be filled as quickly as possible. This favors technologies where model sharding can be performed efficiently (i.e.\ with balanced shards in terms of type-instance relationships), and elementary queries 
can be executed efficiently.

\subsection{Notification mechanisms}

% -- The Rete algorithm defines an asynchronous network of communicating nodes. This is essentially a dataflow network, with two types of nodes. Change notification objects (\emph{tokens}) are propagated to intermediate \emph{worker nodes} that perform operations, like filtering tokens based on constant expressions and performing join or antijoin operations based on their contents. The worker nodes store partial (interim) query results in their own memory. In contrast, \emph{production nodes} are terminators that provide an interface for fetching query results and also their changes (\emph{deltas}). Connections between nodes can be \emph{local} (within one host) or \emph{remote} (when two Rete nodes are allocated to different hosts).

\emph{Model change notifications} are required by incremental query evaluation, thus model changes are captured and their effects propagated in the form of \emph{notification objects} (NOs). The notifications generate \emph{tokens} that keep the Rete network's state consistent with the model. \iqd{}'s middleware layer facilitates notifications by providing a facade for model manipulation operations.

\figref{incqueryd-architecture}

\subsubsection{Notification in distributed database management systems}

While relational databases usually provide \emph{trigger}s for generating notifications, most triplestores and graph databases lack this feature. Among our primary database backends, 4store provides no triggers at all. Titan and Neo4j incorporate Blueprints, which provides an \texttt{EventGraph} class capable of generating notification events, but the events are only propagated in a single JVM (Java Virtual Machine). Implementing distributed notifications would require us to extend the \texttt{EventGraph} class and use a messaging framework. This is subject to future work (see \autoref{sec:future-work}). 

Because the lack of support for distributed notifications, in \iqd{}'s current implementation, notifications are controlled by the middleware. The notification messages are propagated through the Rete network via the Akka messaging framework. 

%\subsection{Incremental queries and change propagation}
%\label{subsec:incrementality}

%Distributed incremental query evaluation introduces a number of challenges. In the following section, we will describe these and present \iqd{}'s solutions for them.






\subsection{Incremental query evaluation}
\label{subsec:incrementality}

%Kicsit más logika alapján csinálnám ezt.
%Én előre venném az inkrementális (gráf)lekérdezés, mint ötlet/technika bemutatását, utána mesélném el a Retét, mint ennek egy megvalósítását, és csak nagyon röviden említeném az egyéb alternatívákat (főleg arra koncentrálva, hogy a Retéhez képest mi a jellegzetes különbségük).

Some queries, e.g.\ well-formedness constraints in MDE are evaluated many times, while the data sets they are evaluated on only changes to a small degree. In these cases, the idea of incremental query evaluation arises naturally: to speed up queries, we should not start the evaluation all over again. Instead, we should rely on the (partial) results derived during the previous executions of the query and process the changes that occured.
 
In practice, incremental query evaluation algorithms typically use data structures for caching the interim results. This  means that they usually consume more memory, in other words, they trade memory consumption for execution speed. This approach, called \emph{space--time tradeoff}, is well-known and widely used in computer science.

In the following, we provide an overview of the Rete algorithm, which forms the theoretical basis of \eiq{} and \iqd{}.

\subsubsection{Rete in general}
\label{subsubsec:rete}

\iqd{} is based on the Rete algorithm, which provides a propagation network for incremental graph pattern matching. The algorithm was originally created by Charles Forgy~\cite{Forgy} for rule-based expert systems. Gábor Bergmann adapted the algorithm for EMF models and added many tweaks and improvements to it~\cite{BergmannRete}.

The Rete algorithm defines an asynchronous network of communicating nodes. This is essentially a dataflow network, with two types of nodes. Change notification objects (\emph{tokens}) are propagated to intermediate \emph{worker nodes} that perform operations, like filtering tokens based on constant expressions and performing join or antijoin operations based on their contents. The worker nodes store partial (interim) query results in their own memory. In contrast, \emph{production nodes} are terminators that provide an interface for fetching query results and also their changes (\emph{deltas}). Connections between nodes can be \emph{local} (within one host) or \emph{remote} (when two Rete nodes are allocated to different hosts).

\subsubsection{Similar algorithms}

Along the original Rete algorithm, many algorithms were developed for incremental pattern matching. Rete itself has improved versions (Rete II, Rete III, Rete-NT), however, unlike the original algorithm, these are not publicly available. 

Other novel algorithms include TREAT \cite{Miranker:1991:OPT:627280.627434}, which aims at minimizing memory using only indexers and dropping partial results, while having the same algorithmic complexity as Rete. Another candidate is the LEAPS \cite{Batory:1994:LA:899216} algorithm, which is claimed to provide better space--time complexity. However, we found that even LEAPS is difficult to understand and implement even on a single workstation, not to mention the distributed case. 

Because the Rete algorithm is well-known and well-understood by the \eiq{} team, we decided to build \iqd{} on the same foundation. Experimenting with improved versions or alternative approaches is subject to future work.

\subsection{\iqd{}'s architecture}

\iqd{}'s architecture consists of three layers: the storage layer, the middleware and the production network. 
The \emph{storage layer} is a distributed database which is responsible for persisting the graph. 
The client application communicates with the \emph{middleware}. The middleware provides a unified API for accessing the database. It also sends change notifications to the production network and retrieves the query results from the production network. 
The \emph{production network} is implemented with a distributed Rete net which provides incremental query evaluation. 

\myFigure{incqueryd-architecture}{\iqd{}'s architecture on a four-node cluster}

The \iqd{} architecture in a four-node cluster configuration is shown in \figref{incqueryd-architecture}.


\subsection{Deployment and configuration}
\label{subsec:deployment-configuration}

Deploying, configuring and operating a distributed pattern matcher is a complex task. In the following, we will break down this task to smaller steps and present our tools for solving them.

\subsubsection{Degrees of freedom}

The deployment and configuration of a distributed pattern matcher involves many degrees of freedom:

\begin{itemize}
  \item We may choose different database implementations due to the \iqd{}'s backend-agnostic nature. Until now, we experimented with property graph databases (Neo4j, Titan) and triplestores (4store).
  \item We may use different database sharding strategies (e.g.\ random partitioners or more sophisticated sharding methods based on domain-specific knowledge).
  \item We may choose different strategies to allocate the Rete nodes. Note that in theory, this is orthogonal to the database's sharding strategy. However, we expect that keeping the Rete network's type indexer nodes and the instances of the given type on the same server would improve the speed of the initialization and modification tasks significantly.
\end{itemize}
 
\subsection{Workflow}
\label{subsec:workflow}

In the following part, we will describe the workflow behind the pattern matching process. Starting from a metamodel, an instance model and a graph pattern, we will cover the problem pieces that need to be solved for setting up an incremental, distributed pattern matcher. The workflow is shown on \figref{recipe}.

\myFigureSmall{recipe}{The workflow of \eiq{} (blue) and \iqd{} (green)}

\subsubsection{Analyze the metamodel and the query}

\paragraph{Task.} First, we determine the constraints defined by the query pattern. The matches satisfying these constraints will define the results of the query.

\paragraph{Implementation.} The pattern is defined in an IQPL (\iq{} Pattern Language) text file. Using Xtext~\cite{Xtext}, an Eclipse-based framework for creating domain-specific languages, the textual representation of the pattern is parsed to an EMF model. Based on the EMF model's pattern and the metamodel, a constraint network called \textit{PSystem} (short for \textit{Pattern System}) is generated. 

\subsubsection{Build a Rete layout}

\paragraph{Task.} To allow incremental query evaluation, we create a Rete net based on the constraints derived from the query.

\paragraph{Implementation.} As we mentioned earlier, we aim to reuse as much of \eiq{}'s existing code base as possible. As part of this attempt, we introduced the concept of \textit{Rete recipe}s which define the layout of a Rete network.    

\subsubsection{Allocate the Rete network in the cloud's nodes} 

\paragraph{Task.} Because of its single workstation nature, \eiq{} simply unfolds the Rete net based on the derived Rete recipe. At the same time, \iqd{} operates in a distributed environment where local resource exhaustion, network latency and throughput are critical aspects. 

\paragraph{Implementation.} Currently, the allocation of the Rete nodes is done manually. To address this limitation, we plan to utilize CSP (Constraint Satisfaction Problem) solvers, or dynamic techniques like DSE (Design Space Exploration)~\cite{DSE11}. 

\subsubsection{Bootstrap the system}

\paragraph{Task.} Based on the Rete network's allocation, we have to deploy the Rete nodes in the distributed systems. After the successful deployment, the Rete network has to be initialized. Due to the Rete algorithm's asynchronous nature, it uses a termination protocol to signal when the data processing is finished. 

\paragraph{Implementation.} In \iqd{}'s prototype, both the bootstrapping and the Rete network's operation is carried out automatically. The Akka actors representing the Rete nodes are deployed and initiated using Akka's \textit{programmatic remote deployment} feature. For signalling the end of data processing, an asynchronous termination protocol was implemented. 


\subsection{Termination protocol}

