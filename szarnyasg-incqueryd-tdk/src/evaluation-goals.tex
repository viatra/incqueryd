\section{Goals}

We implemented a prototype of \iqd{} based on the architecture presented in \autoref{chap:overview}. A working prototype is beneficial for a number of reasons. First, it serves as a proof concept by demonstrating that a distributed, incremental pattern matcher is feasible with the technologies currently available.
On the other hand, it gives us the opportunity to define and run benchmarks, so that we can evaluate the scalability aspects of the system.

\subsection{Dimensions of Scalability}

A distributed system \emph{scalability} has multiple dimensions. Usually, when aiming for \emph{horizontal scalability}, the most emphasized dimension is the \emph{number of processing nodes} (computers) in the system. However, there are other important aspects that include \emph{local resources} of the servers, \emph{network communication overhead}, etc. The main goal of our benchmark was to prove that an \iqd{} cluster is indeed capable of processing queries for large models.
